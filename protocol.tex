
\documentclass[a4paper,10pt,left=1.5cm,right=1.5cm,top=1.5cm,bottom=1.5cm]{article}

\usepackage{geometry}

\geometry{a4paper,left=3cm,right=3cm,top=1cm,bottom=2cm}

\usepackage[utf8x]{inputenc}
\usepackage{url}

\title{1. Praktikum: Modellierung von Informationssystemen}
\author{Andreas Krohn, Benjamin Vetter}

\begin{document}

\maketitle

% Teilbereich 1: Wettbewerbsfähigkeit SUN / Oracle
% Teilbereich 2: Wettbewerbsfähigkeit IBM
% Teilbereich 3: Zukunftsorientierung

\section*{Argumentation für einen Kauf von SAP durch HP (Pro-Seite)}

\subsection*{These 1: Durch eine Übernahme von SAP positioniert sich HP vor Oracle}

Durch die Übernahme von Sun durch Oracle zu Beginn des Jahres hat sich Oracle in das Hardwaregeschäft eingekauft.
Oracle konnte in Anschluss dessen mit 48 Prozent im Vergleich zum Vorjahr deutlich wachsen \cite{oraclegrows} und hat sich in die Lage versetzt seine Software-Lösungen zukünftig als Hardware-Software-Lösungen zu vertreiben.
Daher ist davon auszugehen, dass die enge Partnerschaft von HP und Oracle über kurz oder lang kaputt gehen wird \cite{oraclesun}, da Oracle in das Servergeschäft von HP vordringt.

HP seinerseits kann durch eine Übernahme von SAP mit einem Schlag der weltweit viertgrößte Softwarehersteller \cite{sap} und damit zu einem "Big-Player" der Industrie werden.
Oracle und HP würden durch ihre Übernahmen zu direkten Konkurrenten.
Dabei positioniert sich HP jedoch strategisch geschickt vor Oracle, da SAP im ERP-Bereich Marktführerschaft besitzt \cite{erpmarket}.
Außerdem kann Oracle dem neuen Konkurrenten HP, trotz der Übernahme von Sun, im Hardwarebereich nicht das Wasser reichen.
So hat die Sparc-Architektur kontinuierlich Marktanteile gerade gegen die konkurrierenden Plattformen von HP u.a. verloren \cite{sparc}.

SAP ist durch Anwendungssoftware groß geworden und in dieser verstrickt.
SAP-Vorstandsmitglied Jim Hagemann Snabe erklärt die Strategie von Oracle für falsch, da Kunden offene Systeme wollen und sich nicht an einen Anbieter heften würden.
Der Aktienkurs von Oracle hat sich jedoch in fünf Jahren mehr als verdoppelt, während SAP nur um ein Viertel gewachsen ist.
Die Skepsis der Investoren liegt laut der Frankfurter Allgemeinen Zeitung in der Konzentration auf althergebrachte Software \cite{faz}.
HP kann durch eine Übernahme und einer Modernisierung in Form eines Strategiewandels auch vom Oracle-Erfolgsprinzip profitieren, indem zukünftig integrierte Lösungen angeboten werden.
Dabei kann HP den Vorsprung im Hardwarebereich nutzen um eine HP/SAP-Plattform, 
die heute bereits zu 50\% anzutreffen ist \cite{50percent}, 
weiter auszubauen und Argumente gegen eine Oracle/Sun-Plattform zu schaffen.

\subsection*{These 2: Durch eine Übernahme von SAP wird HP resistenter gegen wirtschaftliche Schwankungen}

Betrachtet man die Geschäftsstrategie von IBM, so stellt man fest, dass sich die strategische Ausrichtung von IBM im Laufe der Zeit gewandelt hat.
Von einem primär auf Hardware ausgerichtetem Portfolio ist IBM heute eher ein Software- als ein Hardwarekonzern \cite{ibmvshp}.
Dank dieser Strategie hat IBM wesentlich weniger unter der Wirtschaftskrise gelitten als HP \cite{ibmvshp}.

HP bedient heute eher den funktionalen Apparat der IT-Landschaft.
SAP operiert hingen im Mangement-Bereich und verkauft seine Produkte an CIOs und darüber \cite{cioandabove}.
Durch einen Zusammenschluss erlangt HP direkten Zugang zu einem sehr lukrativen Geschäftsfeld: den Business-Kunden auf Managementebene.

Diese breitere Aufstellung hat auch für HP u.a. eine gewisse Immunität gegenüber wirtschaftlichen Schwankungen zur Folge.

IBM kann im Augenblick aufgrund der Ausrichtung der Unternehmen nicht als primärer Konkurrent zu HP angesehen werden.
IBM muss vielmehr als Konkurrent zu SAP angesehen werden \cite{ibmvssap}, so dass HP durch SAP wieder in Konkurrenz mit IBM tritt.

%http://www.faz.net/s/Rub4D8A76D29ABA43699D9E59C0413A582C/Doc~E5A112A6F65FE450A84C2BA10475723EF~ATpl~Ecommon~Scontent.html
%
%"SAP ist in der Anwendungssoftware verstrickt. Diese Programme haben den Konzern groß gemacht."
%
%"Die Skepsis der Investoren hat bei Microsoft und SAP dieselbe Ursache: Es ist die Konzentration auf althergebrachte Software."
%
%"Das Vorstandsduo Jim Hagemann Snabe und Bill McDermott versuchen nun das überholte Image über das Thema Mobilität zu erneuern."
%
%"„Oracle ist auf einem Holzweg,“ sagt Snabe mit Verweis auf die Einkaufsorgie des Erzrivalen. Die Kunden wollten offene Systeme und sich nicht an einen Anbieter heften. Die Investoren sehen das bis heute anders. Während SAP das Kerngeschäft vertiefte, haben andere ihr Kerngeschäft deutlich erweitert."
%
%HP könnte diesen Zustand ändern und auch vom Oracle-Erfolgsprinzip profitieren, indem Hardware-Software bundles gestrickt werden.
%
%"Während sich der Kurs des Erzrivalen Oracle in fünf Jahren mehr als verdoppelte, ist der von SAP nur um ein Viertel gewachsen."
%
%"Die gescheiterten Fusionsverhandlungen mit Microsoft und die Äußerungen der vergangenen Wochen zeigen, dass sowohl Hopp als auch Tschira einem Verkauf nicht im Wege stünden."
%
%HP kann mit einer geschickten "Modernisierung" von SAP erreichen, dass SAP als Tochter von SAP stärker hervorgeht als SAP heute ist und zu den Unternehmen aufschießen, die den Wandel bereits erfolgreich geschafft haben.

\subsection*{Microsoft?}
\subsection*{Zukunftsorientierung?}

\begin{thebibliography}{5}
  \bibitem{oraclegrows}
  \newblock \url{http://www.golem.de/1009/78041.html}

  \bibitem{oraclesun}
  \newblock \url{http://www.silicon.de/cio/b2b/0,39038988,41003134-3,00/oracle_sun_folgenschwerste_uebernahme_der_it.htm}

  \bibitem{sap}
  \newblock \url{http://de.wikipedia.org/wiki/SAP}

  \bibitem{erpmarket}
  \newblock \url{http://whatiserp.net/erp-comparison/erp-vendor-evaluation-2010/}

  \bibitem{ibmvshp}
  \newblock \url{http://itmanagement.earthweb.com/columns/article.php/3806331/HP-vs-IBM--Competitors-or-Allies.htm}

  \bibitem{ibmvssap}
  \newblock \url{http://itmanagement.earthweb.com/columns/entad/article.php/3324551/IBM-vs-SAP-Round-X-and-Counting.htm}

  \bibitem{sparc}
  \newblock \url{http://www.computerwoche.de/hardware/data-center-server/1893468/}

  \bibitem{50percent}
  \newblock \url{http://h50146.www5.hp.com/partners/alliance/sap/club/members/si/files/080226_Alexander.pdf}

  \bibitem{cioandabove}
  \newblock \url{http://blogs.gartner.com/martin_reynolds/?p=25&preview=true}

  \bibitem{faz}
  \newblock \url{http://www.faz.net/s/Rub4D8A76D29ABA43699D9E59C0413A582C/Doc~E5A112A6F65FE450A84C2BA10475723EF~ATpl~Ecommon~Scontent.html}
\end{thebibliography}

%http://www.faz.net/s/Rub4D8A76D29ABA43699D9E59C0413A582C/Doc~E5A112A6F65FE450A84C2BA10475723EF~ATpl~Ecommon~Scontent.html
%
%- HP wird durch den Ausbau in SW resistenter gegen wirtschaftliche Schwankungen
%
%- Oracle andersherum: SW-Riese kauft sich ins HW-Segment ein (Sun)
%- Durch Übernahme von Sun durch Oracle wird strategische Partnerschaft mit HP kaputt gehen
%- Durch Übernahme von SAP ist HP Big-Player in SW-Industrie - im ERP Umfeld Marktführer
%- Gut aufgestellt, da
%  - Deutlich vor Oracle, da sowohl im SW- (SAP) als auch im HW-Geschäft (SPARC) vorn
%  - vor IBM
%  - Oracle: HW = Fremdkörper, HP: HW = Kerngeschäft
%- HP: SAP >= CIO, HP <= CIO: Imagegewinn, Rückwirkung auf HW-Geschäft
%
%http://www.zdnet.de/news/wirtschaft_unternehmen_business_oracle_schliesst_sun_uebernahme_ab_story-39001020-41526371-1.htm
%http://www.informationweek.com/news/global-cio/interviews/showArticle.jhtml?articleID=227701035&cid=nl_IW_btl_2010-10-12_html
%http://www.zdnet.de/news/wirtschaft_unternehmen_business_oracle_steigert_umsatz_und_gewinn_im_zweiten_fiskalquartal_story-39001020-41524670-1.htm
%http://www.heise.de/newsticker/meldung/Oracle-uebernimmt-Sun-214120.html
%
%CEO Larry Ellison erwartet, dass Sun nach dem Zusammenschluss mit Oracle seine Marktanteile und Margen wieder erhöhen wird. Sun werde sich auf Geräte für das gehobene Marktsegment konzentrieren und statt einzelnen Komponenten komplette Pakete anbieten. Als Beispiel nannte er Oracles Exadata-Produkte, die Hardware und Software kombinierten und deren Verkaufszahlen im Vergleich zum ersten Fiskalquartal um das Dreifache gestiegen seien. 
%
%Oracle will SW und HW kombiniert anbieten
%http://www.golem.de/1009/78041.html
%
%"Die enge Partnerschaft mit HP wird wahrscheinlich über kurz oder lang kaputt gehen"
%http://www.silicon.de/cio/b2b/0,39038988,41003134-3,00/oracle_sun_folgenschwerste_uebernahme_der_it.htm
%
%Dadurch würde Oracle Marktanteile abknapsen.
%D.h. die anderen versuchen mehr und mehr Marktanteile abzuknapsen, daher ist HP gezwungen, zu expandieren und im SW-Geschäft Fuß zu fassen.
%
%"Jetzt geht der Trend hin zu integrierten Lösungen und wenn sich ein Anwender heute aus dem Lock-in lösen möchte, dann wechselt er einfach die Applikation"
%http://www.silicon.de/cio/b2b/0,39038988,41003134-3,00/oracle_sun_folgenschwerste_uebernahme_der_it.htm
%
%"Die Sparc-Architektur habe im Markt der Unix-Server kontinuierlich Marktanteile gegen die konkurrierenden Plattformen von IBM und Hewlett-Packard verloren" http://www.computerwoche.de/hardware/data-center-server/1893468/
%"HP habe sich bereits von den eigenen Risc-CPUs verabschiedet und sei auf Intel-Kurs gegangen"
%"So schrieb Sun in fünf der vergangenen acht Jahren Verluste." http://www.wiwo.de/unternehmen-maerkte/oracle-und-sun-fragwuerdige-synergien-394530/
%=> HP im gleichen Geschäft, aber bzgl HW besser aufgestellt
%
%"Hardware ist für Oracle immer noch ein Fremdkörper"
%http://www.computerwoche.de/hardware/data-center-server/1893468/
%=> Vorteil für HP
%=> Von Gegenseite wird kommen: SW ist für HP ein Fremdkörper
%
%"Trotz der Übernahme liegt Oracle in diesem Bereich (22 \% Marktanteil, Stand: 2004) hinter dem Konkurrenten SAP (40 \% Marktanteil, Stand: 2004)" http://de.wikipedia.org/wiki/Oracle#cite_note-Produkte-2
%
%Eine Übernahme von SAP durch Oracle ist auszuschlißen http://www.zdnet.de/news/wirtschaft_unternehmen_business_sap_gruender_spekuliert_ueber_kaufinteressenten_story-39001020-41538718-1.htm
%Durch eine Übernahme von SAP durch HP ist man im ERP Umfeld Marktführer und schließt zu Oracle und IBM auf; wenn diese Chance jedoch verpasst würd, und SAP durch bspw. IBM übernommen würde ist der Zug abgefahren.
%- Man stände vor Oracle, da SAP mehr Marktanteile als Oracle im ERP-Umfeld hat und man im HW-Segment (Oracle: SPARC) Kernkompetenzen hat
%- Man steht bzgl. ERP vor IBM
%
%The company is building a portfolio of cloud technologies, but the clear message is on cutting the cost of computing, rather than enabling its customers to deliver new, innovative services.
%http://blogs.gartner.com/martin_reynolds/?p=25&preview=true
%=> SAP bietet neue chancen, indem die Technologien vereint werden, SAP in the cloud, SaaS
%
%First, SAP operates at the business leader level. It sells to the CIO and above. HP, on the other hand, proliferates in the functional parts of IT, with no significant business leader engagement above the CIO.
%http://blogs.gartner.com/martin_reynolds/?p=25&preview=true
%=> größeres spektrum, breiter aufstellen 
%=> Imagechancen für hp, ggf. auch Rückwirkend auf HW-Geschäft positiv
%
%%% Während des Praktikums gesammelt:
%
%\section{These1}
%
%/THESE/ARGUMENTATION/BELEGE
%
%standpunkt: HP auf Angriff/Verteidigung gebürstet
%
%- Zentralisierung der IT-Landschaft
%- bereits stategische allianz http://h20338.www2.hp.com/enterprise/cache/11813-0-0-0-121.html dh. HP liefert häufig die Infrastruktur auf der SAP Lösungen laufen, investitionssicherheit für kunden
%- Software ist bisher nicht der Fokus von HP, Infrastrukturanbieter, SAP = SW => Angebotslücke, Synergieeffekte
%
%**
%
%First, SAP operates at the business leader level. It sells to the CIO and above. HP, on the other hand, proliferates in the functional parts of IT, with no significant business leader engagement above the CIO.
%
%http://blogs.gartner.com/martin_reynolds/?p=25&preview=true
%
%=> größeres spektrum, breiter aufstellen 
%=> Imagechancen für hp
%
%**
%
%The company is building a portfolio of cloud technologies, but the clear message is on cutting the cost of computing, rather than enabling its customers to deliver new, innovative services.
%
%http://blogs.gartner.com/martin_reynolds/?p=25&preview=true
%
%=> SAP bietet neue chancen, indem die Technologien vereint werden, SAP in the cloud, SaaS
%
%**
%
%Dank der Übernahme von SAP hätte HP mit einem Schlag ein ähnliches Portfolio wie Oracle und IBM
%
%http://www.informationweek.com/news/global-cio/interviews/showArticle.jhtml?articleID=227701035&cid=nl_IW_btl_2010-10-12_html
%
%SAP selbst kein HW Angebot => könnte Neukunden abschrecken => dank HP 
%
%**
%
%SAP passt in HPs aktuelle Strategie
%
%- Partnerschaft mit Teradata (Teradata is a leading provider of powerful, enterprise analytic technologies and services that include Data Warehousing, Business Intelligence and CRM) http://h10134.www1.hp.com/insights/alliances/teradata/ (SAP hatte überlegt diese zu kaufen http://searchsap.techtarget.com/news/1372245/Four-things-SAP-must-consider-before-a-Teradata-acquisition?asrc=EM_NLN_9796986&track=NL-137&ad=733844)
%- Partnerschaft mit Informatica (The Data Integration Company – ist ein führender Anbieter für betriebliche Datenintegrationssoftware.) http://h71028.www7.hp.com/enterprise/cache/8156-0-0-0-121.html
%- EDS gekauft (Outsourcing von IT, Business Process Outsourcing Service, IT Transformation Services) http://de.wikipedia.org/wiki/Electronic_Data_Systems) passt zu SaaS-Strategie
%
%**
%
%- ca 50\% aller SAP-Installationen auf HP-Hardware
%- wenn ein anderer "großer" SAP kauft und HW vertreibt, dann ist die wahrscheinlichkeit groß, dass aus den 50\% deutlich weniger werden
%- h50146.www5.hp.com/partners/alliance/sap/.../080226_Alexander.pdf

\end{document}

